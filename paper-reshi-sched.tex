\documentclass[11pt]{article}
\usepackage[margin=1in]{geometry}
% The preceding line is only needed to identify funding in the first footnote. If that is unneeded, please comment it out.
\usepackage{cite}
\usepackage{amsmath,amssymb,amsfonts}
%\usepackage{algorithmic}
\usepackage{graphicx}
\usepackage{textcomp}
\usepackage{xcolor}
\usepackage{url}
\usepackage{hyperref}
\usepackage{numprint}
%\usepackage{ulem}
\usepackage{balance}

\usepackage{algpseudocode}
\usepackage{algorithmicx}
\usepackage{algorithm}
\usepackage{color}
\usepackage{bibentry}
\usepackage{booktabs}
\usepackage{xspace}
\usepackage{tikz}
\usepackage{todonotes}

% user-specified commands
%\newcommand{\R}{\mathbb{R}}
\newcommand{\N}{\mathbb{N}}
%\newcommand{\llminimal}{${\|\cdot\|}_2$-minimal\xspace}
\newcommand{\Pro}[1]{\mathbf{Pr} \left[\,#1\,\right]}
\newcommand{\pro}[1]{\mathbf{Pr} [\,#1\,]}
\newcommand{\Ex}[1]{\mathbb{E} \left[\,#1\,\right]}
\newcommand{\ex}[1]{\mathbb{E} [\,#1\,]}
\newcommand{\hit}{- [\pi]_s (H[v,s]-H[u,s])}

\newcommand{\Oh}{\ensuremath{\mathcal{O}}}


\newcommand{\centre}[1]{z_{#1}}
\newcommand{\centres}{Z}
\newcommand{\degree}{\operatorname{deg}}
\newcommand{\maxdeg}{\operatorname{maxdeg}}
\newcommand{\diam}{\operatorname{diam}}
\newcommand{\res}{\operatorname{res}}
\newcommand{\Res}{\operatorname{Res}}
%\newcommand{\cond}{\operatorname{cond}}
\newcommand{\Cond}{\operatorname{Cond}}
\newcommand{\proxy}{\operatorname{proxy}}
\newcommand{\excond}{\operatorname{ex\_cond}}
\newcommand{\exres}{\operatorname{ex\_res}}
\newcommand{\exalg}{\operatorname{ex\_alg}}
\newcommand{\dist}{\operatorname{dist}}
\newcommand{\cost}{\operatorname{c}}
\newcommand{\ord}{\operatorname{ord}}
\newcommand{\Vor}{\operatorname{Vor}}
\newcommand{\M}{\mathbf{M}}
\newcommand{\LL}{\mathbf{L}}
%\newcommand{\L}{\mathbf{L}}
\newcommand{\RR}{\mathbb{R}}
\newcommand{\f}{\hat{f}}
\newcommand{\e}{\mathbf{e}}
\newcommand{\dhat}{d}
\newcommand{\llminimal}{${\|\cdot\|}_2$-minimal\xspace}
\newcommand{\NP}{$\mathcal{NP}$}
\newcommand{\disjbigcup}{\mathop{\dot{\bigcup}}} 
\newcommand{\disjcup}{\mathop{\dot{\cup}}} 
\newcommand{\bigO}{\mathcal{O}} 
\newcommand{\polylog}{\operatorname{polylog}} 
\newcommand{\dotcup}{\stackrel{.}{\cup}}
\newcommand{\myvec}[1]{\mathbf{#1}}
\newcommand{\vent}[2]{\myvec{#1}[#2]}
\newcommand{\size}[1]{\operatorname{size}(#1)}
\newcommand{\distance}[2]{\operatorname{d_{G_p}}[#1][#2]}
\newcommand{\prefix}[1]{\operatorname{prefix}(#1)}
\newcommand{\clz}[1]{\operatorname{clz}(#1)}
\newcommand*\xor{\oplus}


\DeclareMathOperator{\intraWeight}{\it intraWeight}
\DeclareMathOperator{\interWeight}{\it interWeight}
\DeclareMathOperator{\subtreeVol}{\it subtreeVol}
\DeclareMathOperator{\cutWeight}{\it cutWeight}
\DeclareMathOperator{\conduct}{\it conduct}
\DeclareMathOperator{\inCutSet}{\it inCutSet}

\newcommand{\iec}{\textit{i.\,e.},\xspace}
\newcommand{\ie}{\textit{i.\,e.}\xspace}
\newcommand{\Ie}{\textit{I.\,e.}\xspace}
\newcommand{\egc}{\textit{e.\,g.},\xspace}
\newcommand{\eg}{\textit{e.\,g.}\xspace}
\newcommand{\Eg}{\textit{E.\,g.}\xspace}
\newcommand{\etal}{\textit{et al.}\xspace}
\newcommand{\Wlog}{w.\,l.\,o.\,g.\xspace}
\newcommand{\wrt}{w.\,r.\,t.\xspace}
\newcommand{\cf}{cf.\xspace}
\newcommand{\etc}{etc.\xspace}

\newcommand{\networkit}{\textsc{NetworKit}\xspace}
\newcommand{\mswap}{\textsc{TiMEr}\xspace}
\newcommand{\karma}{\textsc{KarMa}\xspace}
\newcommand{\greedyallc}{\textsc{GreedyAllC}\xspace}
\newcommand{\dibap}{\textsc{DibaP}\xspace}
\newcommand{\pdibap}{\textsc{PDibaP}\xspace}
\newcommand{\bubble}{\textsc{Bubble}\xspace}
\newcommand{\smooth}{\textsc{Smooth}}
\newcommand{\bubfosc}{\textsc{Bubble-FOS/C}\xspace}
\newcommand{\bubfost}{\textsc{Bubble-FOS/T}\xspace}
\newcommand{\metis}{\textsc{METIS}\xspace}
\newcommand{\kmetis}{\textsc{kMeTiS}\xspace}
\newcommand{\parmetis}{\textsc{ParMETIS}\xspace}
\newcommand{\libtopomap}{\textsc{LibTopoMap}\xspace}
\newcommand{\kappart}{\textsc{KaPPa}\xspace}
\newcommand{\kahip}{\textsc{KaHIP}\xspace}
\newcommand{\mapkahip}{\textsc{KaHIP\_map}\xspace}
\newcommand{\parhip}{\textsc{ParHIP}\xspace}
\newcommand{\mpipp}{\textsc{MpiPP}\xspace}
\newcommand{\jostle}{\textsc{Jostle}\xspace}
\newcommand{\zoltan}{\textsc{Zoltan}\xspace}
\newcommand{\parkway}{\textsc{Parkway}\xspace}
\newcommand{\graclus}{\textsc{Graclus}\xspace}
\newcommand{\party}{\textsc{Party}\xspace}
\newcommand{\scotch}{\textsc{Scotch}\xspace}
\newcommand{\ptscotch}{\textsc{PT-Scotch}\xspace}
\newcommand{\tree}{\textsc{Tree-Matching}\xspace}
\newcommand{\topomatch}{\textsc{TopoMatch}\xspace}
\newcommand{\xpulp}{\textsc{xtraPulp}\xspace}
\newcommand{\thrsh}{$\mathtt{thrsh}$\xspace}
\newcommand{\trunccons}{\textsc{TruncCons}\xspace}
\newcommand{\consol}{\texttt{Consolidation}\xspace}
\newcommand{\consols}{\texttt{Consolidations}\xspace}
\newcommand{\asspart}{\texttt{AssignPartition}\xspace}
\newcommand{\assclus}{\texttt{AssignCluster}\xspace}
\newcommand{\asssd}{\texttt{AssignSubdomain}\xspace}
\newcommand{\compcen}{\texttt{ComputeCenters}\xspace}
\newcommand{\initcen}{\textsc{LoadBasedInitialCenters}\xspace}
\newcommand{\NN}{\mbox{\rm I$\!$N}}
\newcommand{\closu}[1]{\overline{#1}}
\newcommand{\djoko}{Djokovi\'{c} relation\xspace}
\newcommand{\djokoRelated}{Djokovi\'{c}\xspace related\xspace}
\newcommand{\subE}[1]{{#1}_{\mathcal{E}}}
\newcommand{\comm}{\operatorname{Co}}

\newcommand{\argmin}{\operatorname{argmin}\xspace}
\newcommand{\argmax}{\operatorname{argmax}\xspace}
\newcommand{\cc}{\operatorname{Coco}\xspace}
\newcommand{\divers}{\operatorname{Div}\xspace}
\newcommand{\ccd}{\operatorname{Coco^+}\xspace}
\newcommand{\dil}{\operatorname{dil}\xspace}
\newcommand{\contract}{\operatorname{contract}\xspace}
\newcommand{\assemble}{\operatorname{assemble}\xspace}
\newcommand{\modulo}{\operatorname{mod}\xspace}

\newcommand{\parent}{\operatorname{parent}\xspace}
\newcommand{\oldParent}{\operatorname{oldParent}\xspace}
\newcommand{\newParent}{\operatorname{newParent}\xspace}
\newcommand{\newParentLabel}{\operatorname{newParentLabel}\xspace}
\newcommand{\prefLabel}{\operatorname{prefLabel}\xspace}

\newcommand{\initial}{\textsc{Identity}\xspace}
\newcommand{\initialM}{\textsc{Identity}_{\textsc{g}} \xspace}
\newcommand{\initialMO}{\textsc{Identity}_{\textsc{n}} \xspace}
\newcommand{\random}{\textsc{Random}\xspace}
\newcommand{\randomM}{\textsc{Random}_{\textsc{g}} \xspace}
\newcommand{\randomMO}{\textsc{Random}_{\textsc{n}} \xspace}
\newcommand{\rcm}{\textsc{RCM}\xspace}
\newcommand{\rcmM}{\textsc{RCM}_{\textsc{g}} \xspace}
\newcommand{\rcmMO}{\textsc{RCM}_{\textsc{n}} \xspace}
\newcommand{\durebi}{\textsc{DRB}\xspace}
\newcommand{\durebiM}{\textsc{DRB}_{\textsc{g}} \xspace}
\newcommand{\durebiMO}{\textsc{DRB}_{\textsc{n}} \xspace}
\newcommand{\greedyall}{\textsc{GreedyAll}\xspace}
\newcommand{\greedyallM}{\textsc{GreedyAll}_{\textsc{g}} \xspace}
\newcommand{\greedyallMO}{\textsc{GreedyAll}_{\textsc{n}} \xspace}
\newcommand{\greedyallcM}{\textsc{GreedyAllC} \xspace}
\newcommand{\greedyallcMO}{\textsc{greedyAllC}_{\textsc{n}} \xspace}
\newcommand{\greedymin}{\textsc{GreedyMin}\xspace}
\newcommand{\greedyminM}{\textsc{GreedyMin}_{\textsc{g}} \xspace}
\newcommand{\greedyminMO}{\textsc{GreedyMin}_{\textsc{n}} \xspace}
\newcommand{\greedyminc}{\textsc{GreedyMinC}\xspace}
\newcommand{\greedymincM}{\textsc{GreedyMinC}_{\textsc{g}} \xspace}
\newcommand{\greedymincMO}{\textsc{GreedyMinC}_{\textsc{n}} \xspace}
\newcommand{\walshawlarge}{\textsc{WalshawLarge}\xspace}
\newcommand{\complexnets}{\textsc{ComplexNets}\xspace}
\newcommand{\gpmetis}{\textsc{gpMetis}\xspace}
\newcommand{\ndmetis}{\textsc{ndMetis}\xspace}

\newcommand{\seacode}{\textsc{SEAmap}\xspace}
\newcommand{\ouralgo}{\textsc{ParHIP\_map}\xspace }


\newcommand{\real}{\textsc{real}\xspace}
\newcommand{\ba}{\textsc{BA}\xspace}
\newcommand{\rhg}{\textsc{RHG}\xspace}
\newcommand{\rmat}{\textsc{Rmat}\xspace}


%% \newcommand{\cone}{\texttt{cSc}\xspace}
%% \newcommand{\ctwo}{\texttt{cId}\xspace}
%% \newcommand{\cthree}{\texttt{cGr}\xspace}
%% \newcommand{\cfour}{\texttt{cLb}\xspace}
\newcommand{\cone}{\mathtt{c1}\xspace}
\newcommand{\ctwo}{\mathtt{c2}\xspace}
\newcommand{\cthree}{\mathtt{c3}\xspace}
\newcommand{\cfour}{\mathtt{c4}\xspace}


\newcommand{\dmax}{D_{\max}}
\newcommand{\algo}[1]{\textsc{#1}}
\newcommand{\new}[1]{{\color{teal}#1}}
\newcommand{\skug}[1]{{\color{blue}[SK: #1]}}
\newcommand{\hmey}[1]{{\color{red}[HM: #1]}}
\newcommand{\AB}[1]{{\color{purple}[AB: #1]}}
\newcommand{\tochange}[1]{{\color{orange}[Change/Check!: #1]}}
\newcommand{\TODO}[1]{\todo[inline]{#1}}


\newcommand{\daghetpart}{\algo{DagHetPart}\xspace}
\newcommand{\reshipart}{\algo{ReshiPart}\xspace}
\newcommand{\reshi}{\algo{Reshi}\xspace}



\newcommand{\bottomlevel}[1]{\underline{l}_{#1}} % underline short italic
\newcommand{\criticalpath}{\mathcal{P}}
\newcommand{\parents}[1]{\,\Pi_{#1}}
\newcommand{\children}[1]{\,C_{#1}}
\newcommand{\cluster}{\,\mathcal{C}}


\pagestyle{plain}

\begin{document}

    \title{Mapping Memory-Constrained Workflows Using a Recommender System}
%
    \maketitle

    \begin{abstract}
    \end{abstract}


%%%%%%%%%%%%%%%%%%%%%%%%%%%%%%%%%%%%%%%%%%%%%%%%%%


    \section{Introduction}
    \label{sec:intro}
%\skug{Budget: 1 page}

%%% CONTEXT %%%


%%% MOTIVATION %%%

%%% CONTRIBUTION %%%
    \subsubsection*{Contributions}
    \subsubsection*{Outline}

%%%%%%%%%%%%%%%%%%%%%%%%%%%%%%%%%%%%%%%%%%%%%%%%%%


    \section{Related Work}
    \label{sec:related-work}


    \section{Model}
    \label{sec:model}

    Table~\ref{tabnotation} summarizes the main notation. % all symbols used in this paper.

    \begin{table}[h!]
        \begin{center}
            \begin{tabular}{rl}
                \hline
                \textbf{Symbol}                       & \textbf{Meaning}                                               \\
                \hline
                $G = (V, E)$                          & Workflow graph, set of vertices (tasks) and edges              \\
                %$v$, $u$ or $t$ & One vertex in the graph  \\
                % $E$, $e$ & Set of edges (dependencies) in the graph, one edge   \\
                %$\olsi{v}$, $\ulsi{v}$ & Parents of a task $v$, children of a task $v$ \\
                $\parents{u}$, $\children{u}$         & Parents of a task $u$, children of a task $u$                  \\
                $m_u$                                 & Memory weight of task $u$                                      \\
                $w_u$                                 & Workload of task $u$  (normalized execution time)                 \\
                $c_{u,v}$                             & Communication volume along the edge $(u,v)\in E$               \\
                $F$, $\mathcal{F}$                    & A partitioning function and the partition it creates           \\
                $V_i$                                 & Block number $i$                                               \\ %\wrt~some $F$   \\
                $\cluster$, $k$                    & Compute system and the number of processors in it                 \\
                $p_j$, proc($V_i$)                          & Processor number $j$, processor of block $V_i$                       \\
                $M_j$, $s_j$                               & Memory size and speed of processor $p_j$                                \\
                $\beta$                     & Bandwidth in the compute system                                      \\
                $\bottomlevel{u}$                      & Bottom weight of task $u$       \\
                $\mu_G$, $\mu_i$ & Makespan of the entire workflow $G$ and of a block $V_i$                     \\
                $\Gamma = (\mathcal{V}, \mathcal{E})$                              & Quotient graph, its vertices and its edges                            \\
                %$\nu_i$ & Single vertex in the quotient graph  \\
                $r_u$, $r_{V_i}$                            & Memory requirement of task $u$ and of block $V_i$                       \\
                $r_{\max}$                   & Maximum memory requirement in a workflow                        \\
                %    $\alpha$, $\omega$& Source and target tasks in the workflow   \\
                $\criticalpath$        & Critical path in a workflow \\
                $R = [{V_i, [p_1\cdots p_j]}]$        & \reshi recommendations, per block. $p_1$ is the most suitable. \\
                \hline
                %\multicolumn{4}{l}{$^{\mathrm{a}}$Sample of a Table footnote.}
            \end{tabular}
        \end{center}
        \caption{Notation} \label{tabnotation}
    \end{table}


    \section{Heuristics} %Partitioning-based heuristic}
    \label{sec:heuristics}

    \subsection{\reshi}

    TODO: describe Reshi shortly here

    \subsection{Partitioning-based heuristic -- \reshipart}
    \label{sec.daghetpart}

    We now describe out heuristic \reshipart.
    It uses partitioning to obtain blocks and recommendations from \reshi to help choose optimal processors for the blocks.


    \subsection*{Step 1: Partitioning}

    We adopt this step from \daghetpart.
    We run a DAG partitioner on the input workflow.
    The partitioner takes no heterogeneity in account, and partitions with the aim of minimizing the edge cut.

    %this paragraph is copied from the ipdps paper
    This step returns the initial partitioning function in the form of an array of block numbers.
    For instance, with $n=5$ tasks and $k=2$ processors, we may obtain a result [1,1,1,2,2]
    meaning that tasks 1,2 and 3 are in the same block~$V_1$, and tasks 4 and 5 are
    in another block~$V_2$.

    \subsection*{Step 2: Assignment}

    This step starts with the partitioning function obtained from Step~1.
    In \daghetpart, we map biggest blocks to nodes with biggest memories, taking into account only memory heterogeneity in the
    execution environment.
    However, several nodes can fit memory-wise - and we can make an educated decision when choosing between them.
    %In this case, \daghetpart did not distinguish between better- and worse-fitting nodes.

    \reshipart uses the recommender system \reshi to decide what nodes fit best for each block.
    Algorithm~\ref{alg:ReshiAssign} describes the procedure.

    We first compute the quotient tree $\Gamma$ and its \textit{estimated} critical path $\criticalpath$~(Line~\ref{line:cp}).
    The critical path can only be estimated, because without an assignment to the processors, exact execution times of tasks are unknown.
    We also receive the recommended processors for each block from \reshi - stored as $R$~(Line~\ref{line:r}).

    We start with blocks on the critical path, because assigning them well has an impact on the overall makespan.
    We insert all still unassigned blocks on critical path into a priority queue~$Q$ -- with the required block memory
    size as priority~(Line~\ref{line:pq}).
    Then, going from the largest to the smallest, we try assigning each block~(Line~\ref{line:maxblock}) to one of the
    processors that were recommended for it~(Line~\ref{line:recom-proc}).
    We start from the processor that \reshi recommends the most and move along the recommended processors until we find a free one~(Line~\ref{line:nobusy}).

    If the processor's memory size is big enough to hold this block, then we assign the block there.
    Otherwise, we further partition the block, reinsert the newly created blocks into the quotient graph, recompute the
    critical path and retrieve the recommendations from \reshi~(Line~\ref{line:notfit}).
    We retrieve recommendations for all the blocks, not only the newly created ones, because with the new partitioning,
    the recommendations for the old blocks can change.

    If the block could not have been assigned to any suitable processor, we further partition it down to the memory size
    of the smallest processor~(Line~\ref{line:partdown}).
    After the critical path, we repeat the same procedure with the clocks that lie outside of critical path~(Lines~\ref{line:notcritical}-\ref{line:end}).



    \begin{algorithm}[tb]
        \caption{\algo{ReshiAssign}}
        \label{alg:ReshiAssign}
        \begin{algorithmic}[1]
            \Procedure{\algo{ReshiAssign}}{$G$,$\mathcal{F}$, $\cluster$}
                \Comment{Input: the workflow graph $G$ initial partition $\mathcal{F}$, cluster $\cluster$}

                \State $R\gets$\reshi($F, \cluster$)\label{line:r}
                \Comment Obtain \reshi`s recommendations

                \State $\Gamma \gets $ \algo{QuotientDAG}($G$, $\mathcal{F}$ )\Comment We build a quotient DAG from the blocks of $G$ \label{line:gamma}
                \State $\criticalpath \gets $ \algo{CriticalPath}($\Gamma$)\label{line:cp}
                \Comment Without assignment to processors, we compute the \textit{estimated} critical path
                \State $A \gets \emptyset$;\Comment Assigned blocks

                \For{$\nu \in \criticalpath \setminus A$ }
                    \Comment We start assigning blocks on the critical path
                    \State \texttt{Init} PQ $Q$ with $\criticalpath \setminus A $;\label{line:pq}
                    \Comment{max-priority queue of the blocks by size}
                    \While{\textbf{not} $Q$\texttt{.empty()}}
                        \label{line:whilebig}
                        \State $V_{m} \gets Q$\texttt{.extractMax()};\label{line:maxblock}
                        \For{$p_i \in R[V_m]$} \label{line:recom-proc}
                            \If{$p_i$ is not busy}\label{line:nobusy}
                                \Comment TODO: express busy precisely
                                \If{$r_{V_m} \leq M_{i}$}
                                    \State proc($V_m$) $\gets p_i$;\label{line:schedule-if-fits} $A\gets A \cup {V_m}$
                                \Else \label{line:notfit}
                                    \State $(V_{m_1}, V_{m_2}, \cdots) \gets $ \algo{Partition}($V_m$, 2); \label{line:repartition}
                                    \State $\Gamma \gets $ \algo{QuotientDAG}($G$, $\mathcal{F} \cup (V_{m_1}, V_{m_2}, \cdots)$ )
                                    \State $\criticalpath \gets $ \algo{CriticalPath}($\Gamma$)
                                    \State $R\gets$\reshi($\mathcal{F} \cup (V_{m_1}, V_{m_2}, \cdots), \cluster$)
                                    \State break;
                                \EndIf
                            \EndIf
                        \EndFor
                        \If{$proc(V_m) = NULL$} \Comment The block has not been assigned, partition it down to the memory of the smallest proc.
                        \label{line:partdown}
                            \State  $p_{m} \gets {\min}M(\cluster)$;
                            \While{$r_{V_m} \geq M_m$}
                                \State $(V_{m_1}, V_{m_2}, \cdots) \gets $ \algo{Partition}($V_m$, 2); \label{line:repartitionToSmallest}
                                \State $V_m \gets V_{m_1}$
                            \EndWhile

                        \EndIf
                    \EndWhile
                \EndFor

                \For{$\nu \in \mathcal{V} \setminus \criticalpath \setminus A $} \label{line:notcritical}
                    \State \texttt{Init} PQ $Q$ with $\mathcal{V} \setminus \criticalpath \setminus A$;
                    \Comment{max-priority queue of the blocks by size}
                    \While{\textbf{not} $Q$\texttt{.empty()}}
                        \State $V_{m} \gets Q$\texttt{.extractMax()};
                        \For{$p_i \in R[V_m]$}
                            \If{$p_i$ is not busy}
                                \Comment TODO: express busy precisely
                                \If{$r_{V_m} \leq M_{i}$}
                                    \State proc($V_m$) $\gets p_i$; $A\gets A \cup {V_m}$
                                \Else
                                    \State $(V_{m_1}, V_{m_2}, \cdots) \gets $ \algo{Partition}($V_m$, 2);
                                    \State $\Gamma \gets $ \algo{QuotientDAG}($G$, $\mathcal{F} \cup (V_{m_1}, V_{m_2}, \cdots)$ )
                                    \State $\criticalpath \gets $ \algo{CriticalPath}($\Gamma$)
                                    \State $R\gets$\reshi($\mathcal{F} \cup (V_{m_1}, V_{m_2}, \cdots), \cluster$)
                                \EndIf
                            \EndIf
                        \EndFor
                        \If{$proc(V_m) = NULL$} \Comment The block has not been assigned, partition it down to the memory of the smallest proc.
                        \State  $p_{m} \gets {\min}M(\cluster)$;
                        \While{$r_{V_m} \geq M_m$}
                            \State $(V_{m_1}, V_{m_2}, \cdots) \gets $ \algo{Partition}($V_m$, 2);
                            \State $V_m \gets V_{m_1}$
                        \EndWhile

                        \EndIf
                    \EndWhile
                \EndFor \label{line:end}
            \EndProcedure
        \end{algorithmic}
    \end{algorithm}

    \subsection*{Step 3: Merging}

    We merge like in \daghetpart.

    \subsection*{Step 4: Swaps}

    By this step, every block has to be assigned to a processor (unless the algorithm had failed).
    Therefore, the input is the quotient graph, the execution environment and the recommendations provided by \reshi.

    For each vertex $\nu$ in the quotient graph $\Gamma$, we find the processor that is its best fit according to \reshi.
    However, there is another quotient vertex $\nu'$ assigned to it.
    We then swap $\nu$ and $\nu'$ if such a swap is feasible and note the makespan it results in.
    If the swap improves the makespan, we execute it and continue with the next vertex.


    \begin{algorithm}[b!]
        \caption{Swap}
        \label{alg:ReshiSwap}
        \begin{algorithmic}[1]
            \Function{\algo{swap}}{$\Gamma$, $\cluster$, $R$}
                \Comment Quotient DAG $\Gamma$ whose tasks $\mathcal{V}$ have been assigned to processors, cluster $\cluster$, \reshi recommendations $R$.
                \State $best \gets$ \texttt{pair}($\Gamma$, \texttt{makespan}$(\Gamma)$); \label{line:init-best}

                \State $curr \gets best$;
                \While{true}


                \For{ $(\nu \in \mathcal{V}$}
                    \State $\nu' \gets R[\nu][0]$ \Comment $nu'$ is the quotient vertex assigned to the processor that, according to \reshi, best fits $\nu$
                    \If{\texttt{isSwapFeasible}($best.first$,$\nu$,$\nu'$)}
                        \label{line:feasible-swaps}
                        \State $\Gamma' \gets$ \texttt{swap}$(best.first, (\nu, \nu'))$; \label{line:swap}
                        \State $next \gets$ \texttt{pair}($\Gamma'$, \algo{makespan}$(\Gamma')$);
                        \If{$next.second < curr.second$}
                            \State $curr \gets next$; \label{line:improvement-found}\\
                            \Comment{better solution is stored}
                        \EndIf
                    \EndIf
                \EndFor \label{line:forall-pairs-end}
                \If{$curr.second < best.second$}
                    \State $best \gets curr$; $\Gamma \gets  best.first$;  \label{line:new-best}\\
                    \Comment{execute improving swap and work further with the resulting quotient graph}
                \Else
                % \State
                    ~\textbf{return} $best$; \label{line:return}\\
                    \Comment{stop because no improving swap exists}
                \EndIf
                \EndWhile

            \EndFunction
        \end{algorithmic}
    \end{algorithm}


    \section{Experimental Evaluation}
    \label{sec:expe}

    \subsection{Results}
    \label{sec:results}

    \subsubsection{Running Times}

    \subsubsection{Summary}


    \section{Conclusion}
    \label{sec:conc}



    \bibliography{references}
    \bibliographystyle{abbrv}
\end{document}